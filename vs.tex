% !TEX encoding = UTF-8 Unicode
\documentclass[a4paper]{article}

\usepackage{color}
\usepackage{url}
\usepackage[T2A]{fontenc} % enable Cyrillic fonts
\usepackage[utf8]{inputenc} % make weird characters work
\usepackage{graphicx};

\documentclass{article}
\usepackage{tikz}
\usetikzlibrary{positioning}
\usetikzlibrary{calc}


\usepackage[english,serbian]{babel}
%\usepackage[english,serbianc]{babel} %ukljuciti babel sa ovim opcijama, umesto gornjim, ukoliko se koristi cirilica

\usepackage[unicode]{hyperref}
\hypersetup{colorlinks,citecolor=green,filecolor=green,linkcolor=blue,urlcolor=blue}

\usepackage{listings}

%\newtheorem{primer}{Пример}[section] %ćirilični primer
\newtheorem{primer}{Primer}[section]

\definecolor{mygreen}{rgb}{0,0.6,0}
\definecolor{mygray}{rgb}{0.5,0.5,0.5}
\definecolor{mymauve}{rgb}{0.58,0,0.82}

\lstset{ 
  backgroundcolor=\color{white},   % choose the background color; you must add \usepackage{color} or \usepackage{xcolor}; should come as last argument
  basicstyle=\scriptsize\ttfamily,        % the size of the fonts that are used for the code
  breakatwhitespace=false,         % sets if automatic breaks should only happen at whitespace
  breaklines=true,                 % sets automatic line breaking
  captionpos=b,                    % sets the caption-position to bottom
  commentstyle=\color{mygreen},    % comment style
  deletekeywords={...},            % if you want to delete keywords from the given language
  escapeinside={\%*}{*)},          % if you want to add LaTeX within your code
  extendedchars=true,              % lets you use non-ASCII characters; for 8-bits encodings only, does not work with UTF-8
  firstnumber=1000,                % start line enumeration with line 1000
  frame=single,	                   % adds a frame around the code
  keepspaces=true,                 % keeps spaces in text, useful for keeping indentation of code (possibly needs columns=flexible)
  keywordstyle=\color{blue},       % keyword style
  language=Python,                 % the language of the code
  morekeywords={*,...},            % if you want to add more keywords to the set
  numbers=left,                    % where to put the line-numbers; possible values are (none, left, right)
  numbersep=5pt,                   % how far the line-numbers are from the code
  numberstyle=\tiny\color{mygray}, % the style that is used for the line-numbers
  rulecolor=\color{black},         % if not set, the frame-color may be changed on line-breaks within not-black text (e.g. comments (green here))
  showspaces=false,                % show spaces everywhere adding particular underscores; it overrides 'showstringspaces'
  showstringspaces=false,          % underline spaces within strings only
  showtabs=false,                  % show tabs within strings adding particular underscores
  stepnumber=2,                    % the step between two line-numbers. If it's 1, each line will be numbered
  stringstyle=\color{mymauve},     % string literal style
  tabsize=2,	                   % sets default tabsize to 2 spaces
  title=\lstname                   % show the filename of files included with \lstinputlisting; also try caption instead of title
}

\tikzset{relation/.style={%
           shape=diamond,draw=relation@colour!50!gray,
           ultra thick,fill=relation@colour!25!white,
           minimum height=2em},
         entity/.style={%
           shape=rectangle,draw=entity@colour!50!gray,
           ultra thick,fill=entity@colour!25!white,
           minimum height=2em}}

\begin{document}

\title{Naslov seminarskog rada\\ \small{Seminarski rad u okviru kursa\\Metodologija stručnog i naučnog rada\\ Matematički fakultet}}

\author{Prvi autor, drugi autor, treći autor, četvrti autor\\ kontakt email prvog, drugog, trećeg, četvrtog autora}

%\date{9.~april 2015.}

\maketitle

\listoffigures



% 1)
% 1. max = a;
% 2. if (max < b) 
% 3. { max = b; }
% 4. if (max < c) 
% 5. { max = c; }
% 6. printf("%d\n", max);


% https://tex.stackexchange.com/questions/251642/draw-arrows-between-nodes-with-tikz
% node distance=2cm

\usetikzlibrary{er}

\begin{tikzpicture}[
shorten >=1pt,
if_statement/.style={relationship, draw=green!60, fill=green!5, very thick, minimum size=0.2},
command/.style={entity, draw=red!60, fill=red!5, very thick, minimum width=3cm,minimum height=1cm},
]
%Nodes
\node[command] (1) {max = a;};
\node[if_statement] (2) [below=of 1] {if (max < b)};
\node[command] (3) [below=of 2] { max = b;};
\node[if_statement] (4) [below=of 3] {if (max < c)};
\node (2_ne) [right=of 3] {ne};
\node (2_da) at ($(2)!0.6!(3)$) {da};
\node[command] (5) [below=of 4] {max = c;};
\node[command] (6)  [below=of 5] { printf("\%d\n", max); };
\node (4_ne) [left=of 5] {ne};
\node (4_da) at ($(4)!0.6!(5)$) {da};


%Lines
\draw[->] (1) -- (2);
\draw[->] (2) -| (2_ne) |- (4);
\draw[->] (2) -- (2_da) -- (3);
\draw[->] (3) -- (4);
\draw[->] (4) -| (4_ne) |- (6);
\draw[->] (4) -- (4_da) -- (5);
\draw[->] (5) -- (6);
%\draw[->] (2) -- +(4,0) -- +(4,0) node[midway, right] |- (4); 
%\draw[->] (2) -| + (4,0) (3.east) + (4,0) node[right]  {edge label 8} |- (4); 
%\node [right,anchor=west]  at  {edge label 8} ;   

%\path[->] (2.west)  edge node {0}  (4.west) 
% https://tikz.dev/tikz-arrows

%\draw[->]  (2.west) .. controls + (left:30mm) and +(down:7mm) .. (4.west);

%\draw[->] (1) -- (2) {true};
%\draw[->] (2) -- (3);
%\draw[->] (3) .. controls +(down:17mm) and +(right:7mm) .. (4);
\end{tikzpicture}



%\begin{tikzpicture}[y=1.5cm]
%
%\begin{scope} [every node/.style={draw,circle}]
%    \node (n4) at (2, -3) {4};
%    \node (n5) at (2, -4) {5};
%    \node (n6) at (1, -5) {6};
%    \node (n7) at (3, -5) {7};
%    \node (n8) at (1, -6) {8};
%    \node (n9) at (3, -6) {9};
%    \node (n10)at (2, -7) {10}; 
%\end{scope}
%
%\begin{scope}[->,every node/.style={ right}]
%  \draw (n4)-- node{lengthy edge label 1} (n5);
%  \draw (n5)-- node[left] {lengthy edge label 2}(n6)  ;
%  \draw (n5)-- node (c) {lengthy edge label 3}(n7) ;
%  \draw (n6)-- node[left] {edge label 4}(n8)  ;
%  \draw (n7)-- node {edge label 5}(n9)  ;
%  \draw (n9)-- node {label 6}(n10) ;
%  \draw (n8)-- node[left] {label 7}(n10) ;
%\end{scope}
%
%\draw[->] (n10) -| (c.east) coordinate (d)  |-  (n4);
%\node [right,anchor=west]  at (n7-|d) {edge label 8} ;   

%\end{tikzpicture}

\end{document}
